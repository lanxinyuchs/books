\documentclass{ericsson}
\usepackage{pdfpages}
\usepackage{doxygen}
\usepackage{fancyhdr}
\usepackage{hyperref}
\hypersetup{
    colorlinks=true,
    citecolor=black,
    filecolor=black,
    linkcolor=blue,
    urlcolor=blue
}

% Set header fields

% The document name (type). This shall always be written in
% capital letters. See document 1010-105 for exact instructions 
% on how to use this field. Examples: 'INFORMATION', 'DESCRIPTION'
% 'NOTICE', 'MINUTES' etc. The document name shall match the
% decimal class of the document!
\doctype{INTERWORKING DESCRIPTION}

% Field for handling (0034-116 sec 8.3) is currently not
% supported.

% Security classification, choose from:
% 'Open', 'Ericssonwide Internal', 'Limited Internal', 'Confidential',
% 'Strictly Confidential', 'TOP SECRET', 'CLASSIFIED'
% See document 0034-116 section 8.4 for exact instructions
% on how to use this field.
\infotype{Ericsson Internal}

% The signum and name of the author of the document.
% See document 0034-116 section 7.3 for exact instructions
% on how to use this field.
\docprep{UABKIJU Kimmo Juuj�rvi}

% The signum and name of the document responsible (owner) or
% approver for the document. For documents in decimal notation
% this is a 'document responsible' whereas for office notation
% documents it is and 'approver'.
% See document 0034-116 section 7.4 and section 7.6 for exact 
% instructions on how to use this field.
\docresp{}

% Document number:
% *  For product documents you shall use the ABC-class and a decimal
%    class, see document 0012-078.
% *  For department documents use your department code and a number 
%    on the form AB/CD-06:001 where ``06'' is a year and ``001'' a 
%    running number. See document 0012-088.
% *  For locally managed documents you can use your corporate 
%    signature with two-digit year and a running number like this. 
%    (See 0012-0088 section 2.2.3.)
%
% A language code MUST follow the document number. Use for example:
% 'Uen' = United Kingdom English
% 'Uae' = American English
% 'Usv' = Swedish
% 'Ucf' = Canadian French
% 'Ux' = More than one language or language independent
% All available language codes are listed in document 0012-104.
%\docno{XXXXXXXXXXXXXXXXXXX}
\docno{15519-CNX 901 3303 Uen}
% The corporate signature of a person who checked this document for quality,
% if deemed necessary. See document 0034-116 section 7.5 for exact 
% instructions on how to use this field.
\docchk{}

% Ericsson revision style: PA1, PA2, PA3 .. -> A
% then PB1, PB2, PB3 .. -> B etc.
% See document 1092-211 for exact instructions on how to use this
% field.
\docrev{PA1}

% Override the day of the current date by setting it hard
% with this macro if need be. See document 0034-116 section
% 8.6 for exact instructions on how to use this field.
% \docdate{2006-09-08}

% The 'File' or 'Belongs to' or 'Refers to' field is the
% larger collection of which this document is part. It
% IS NOT THE COMPUTER FILE which the document was generated
% from. See document 0034-116 section 8.10 for an exact 
% instruction on how to use this field.
\file{}

%%%%%%%%%%%%%%%%%%%%%%%%%%%%%%%%%%%%%%%%%%%%%%%%%%%%%%%%%%%%%%%%%%%%%
%%
%% END OF HEADERS AND BEGINNING OF DOCUMENT.
%%
%%%%%%%%%%%%%%%%%%%%%%%%%%%%%%%%%%%%%%%%%%%%%%%%%%%%%%%%%%%%%%%%%%%%%

\begin{document}
\vspace*{20mm}
\doctitle{ITCLNH Link Configuration Interfaces}

\tableofcontents

\newpage

\section{Introduction}

\subsection{About this document}

This document describes the link configuration interfaces provided by ITCLNH.

\subsection{Revision history}

\begin{dochistory}
PA1 & 20140610 & uabkiju: First version\\
\end{dochistory}

\section{Basic Concepts}

The user of these interfaces should be familiar with ITC. 
The provided interfaces allow a user to establish an ITC connection for 
locating a remote mailbox and exchanging messages between a local and a 
remote mailbox.

\section{LINK Configuration Interfaces}

ITCLNH provides the following interfaces:

\begin{itemize}
\item ECBC - ECBus Link Configuration interface.
\item EOLC - Emca OAM Link Configuration interface.
\item ROLC - Radio OAM Link Configuration interface.
\end{itemize}

\subsection{ECBC}

ITCLNH implements an ITC communication link on the EC bus (Equipment Control Bus). The 
transport layer is an implementation of HDLC unbalanced operation, normal reponse mode. To establish the
connection one side should be set up as a primary station and the other as a secondary station. 
See \hyperref[doxygen/ecb-link-api_8h]{ECBC interface description} for details.

\subsection{EOLC}

ITCLNH implements an ITC communication link between the DU host processor and the emca units.
The transport is an implementation on top of EBCOM. See \hyperref[doxygen/eolc-link-api_8h]
{ROLC interface description} for details.

\subsection{ROLC}

ITCLNH implements an ITC communication link between the DU host processor and the radio units.
The transport is an implementation of unbalanced Asynchronous Response Mode HDLC over the CPRI link. 
See \hyperref[doxygen/rolc-link-api_8h] {ROLC interface description} for details.


\section{Terminology}

\begin{description}
\item[-]
\end{description}

\section{References}

\begin{description}
\item[ITC - Inter Thread Communication]
\end{description}


\newpage
\appendix
%\includepdf[pages=3-, addtotoc={3, section, 1, Whatever, filedox}]{doxygen/refman.pdf}
%\section{\\File Documentation} \label{App:AppendixA}
\include{doxygen/ecb-link-api_8h}
\include{doxygen/eolc-link-api_8h}
\include{doxygen/rolc-link-api_8h}

\end{document}

